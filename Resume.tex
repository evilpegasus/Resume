%%%%%%%%%%%%%%%%%%%%%%%%%%%%%%%%%%%%%%%%%
% Medium Length Professional CV
% LaTeX Template
% Version 2.0 (8/5/13)
%
% This template has been downloaded from:
% http://www.LaTeXTemplates.com
%
% Original author:
% Trey Hunner (http://www.treyhunner.com/)
%
% Important note:
% This template requires the resume.cls file to be in the same directory as the
% .tex file. The resume.cls file provides the resume style used for structuring the
% document.
%
%%%%%%%%%%%%%%%%%%%%%%%%%%%%%%%%%%%%%%%%%

%----------------------------------------------------------------------------------------
%	PACKAGES AND OTHER DOCUMENT CONFIGURATIONS
%----------------------------------------------------------------------------------------

\documentclass{resume} % Use the custom resume.cls style

\usepackage[left=0.75in,top=0.6in,right=0.75in,bottom=0.6in]{geometry} % Document margins
\usepackage{hyperref}
\usepackage{xcolor}
\definecolor{darkblue}{rgb}{0.04,0,0.5}
\hypersetup{colorlinks,breaklinks,linkcolor=darkblue,urlcolor=darkblue,anchorcolor=darkblue,citecolor=darkblue}

\name{Ming Fong} % Your name
\address{\href{mailto:mingfong@berkeley.edu}{mingfong@berkeley.edu}} % Your phone number and email
\address{\href{https://www.linkedin.com/in/mingfong/}{linkedin.com/in/mingfong}} % Your secondary addess (optional)

\begin{document}

%----------------------------------------------------------------------------------------
%	WORK EXPERIENCE SECTION
%----------------------------------------------------------------------------------------

\begin{rSection}{Experience}

\begin{rSubsection}{Microsoft Corporation}{June 2019 - August 2019}{Software Engineering Intern}{Redmond, WA}
\item High School Intern on the Azure Core OS and Intelligent Edge (Windows) team
\item Developed a Windows Presentation Foundation (WPF) application by implementing C\# and XAML
\item Set up SQL database tables with relevant queries and REST APIs
\item Used agile methodologies with a small team to coordinate workflow and iterative development
\end{rSubsection}

%------------------------------------------------

\begin{rSubsection}{Kumon North America, Inc.}{October 2018 - June 2019}{Center Assistant}{Renton, WA}
\item Tutored students grades 6-12 in Calculus, Algebra, and English writing
\item Graded and annotated students' classwork and homework
\end{rSubsection}

\end{rSection}

%----------------------------------------------------------------------------------------
%	EDUCATION SECTION
%----------------------------------------------------------------------------------------

\begin{rSection}{Education}
    \begin{rSubsection}{University of California, Berkeley}{June 2020 - May 2024}{Bachelor of Arts, Physics and Computer Science}{Berkeley, CA}
        Freshman student majoring in Physics and Computer Science
        \end{rSubsection}

{\bf Hazen Senior High School} \hfill {September 2016 - June 2020}\\
{\bf Cumulative GPA:} 4.0 {\bf Rank:} 1/383 \smallskip \\
{\bf Test Scores:} SAT: 1560, SAT Math II: 800, SAT Physics: 800, Advance Placement: 5 on all 12 exams taken \\
{\bf Activities:} Math Club (President), Earth Corps (President), Table Tennis Club (Founder, President)\\
{\bf Awards:} Math Departmental Achievement Award, Rotary Youth of the Month, YMCA Leadership Award (x2)
\end{rSection}

%----------------------------------------------------------------------------------------
%	EDUCATION SECTION
%----------------------------------------------------------------------------------------

\begin{rSection}{Projects}

\begin{rSubsection}{\href{https://www.kaggle.com/c/halite/leaderboard}{Halite AI Programming Challenge by Two Sigma}}{June 2020 - August 2020}{}{}
\item Ranked in the top 5\% (33/818) of all submissions
\item Developed a Python AI to complete in the Halite IV simulation environment
\end{rSubsection}

\begin{rSubsection}{\href{https://yearbook-hhs.web.app/}{Yearbok 2020}}{June 2020 - July 2020}{}{}
\item Developed a web app for students and graduates to virtually sign yearbooks
\item Created with HTML/CSS/Javascript and Google Firebase for hosting and backend
\end{rSubsection}

\begin{rSubsection}{\href{https://github.com/microsoft/QuantumKatas}{Quantum Katas}}{July 2019 - August 2019}{}{}
\item Open source contributor to the Quantum Computing Tutorials team during Microsoft's 2019 Hackathon
\item Wrote a Q\# task for learning quantum superposition with integration into Jupyter Notebook and C\#
\end{rSubsection}

\end{rSection}

%----------------------------------------------------------------------------------------
%	TECHNICAL STRENGTHS SECTION
%----------------------------------------------------------------------------------------

\begin{rSection}{Skills}

\begin{tabular}{ @{} >{\bfseries}l @{\hspace{6ex}} l }
Software Languages & Python, Java, C\#, SQL, HTML/CSS/JavaScript \\
Tools & Jupyter Notebook, Visual Studio, Eclipse, Git \\
Languages & English, Mandarin, Cantonese, German \\
\end{tabular}

\end{rSection}

%----------------------------------------------------------------------------------------
%	EXAMPLE SECTION
%----------------------------------------------------------------------------------------

%\begin{rSection}{Section Name}

%Section content\ldots

%\end{rSection}

%----------------------------------------------------------------------------------------

\end{document}
